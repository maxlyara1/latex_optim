
\documentclass[a4paper]{article}
\usepackage[utf8]{inputenc}
\usepackage{ulem}
\usepackage{upgreek}
\usepackage[T2A]{fontenc}
\usepackage[english,russian]{babel}
\usepackage[left=25mm, top=20mm, right=25mm, bottom=30mm, nohead, nofoot]{geometry}
\usepackage{amsmath,amsfonts,amssymb} % математический пакет
\usepackage{fancybox,fancyhdr}
\usepackage{graphicx}%Вставка картинок правильная
\usepackage{float}%"Плавающие" картинки
\usepackage{wrapfig}%Обтекание фигур (таблиц, картинок и прочего)
\usepackage{xcolor}
\usepackage{hyperref}
\hypersetup{colorlinks=true, allcolors=[RGB]{010,090,200}} % цвет ссылок
\newtheorem{theorem}{Теорема}[section]
\newtheorem{corollary}{Corollary}[theorem]
\newtheorem{lemma}[theorem]{Лемма}

\newcommand{\lr}[1]{\left({#1}\right)}

\begin{document}
\Large
\section*{Лекция №6}
\subsection*{Доказательство теоремы Куна-Таккера}
\subsection*{2. Достаточность}
Пусть в точке $x^0 = (x_1^0, ..., x_n^0)$ выполняются условия 1-3 теоремы Куна-Таккера. \\
Имеем выпуклые функции $"f, g_1, ..., g_n"$ и выпуклую допустимую область $\Rightarrow$ \\
$\Rightarrow$ Функция Лагранжа \\
$L = f + \lambda_1g_1 + ... + \lambda_m g_m$ $(\lambda_1 \geqslant 0, ..., \lambda_m \geqslant 0)$ также выпукла (см. св-во 2 выпуклых ф-ий) \\
Если в точке $x^0 = = (x_1^0, ..., x_n^0)$ выполняется условие 1 теоремы Куна-Таккера
$\frac{\partial L}{\partial x_1} = ... = \frac{\partial L}{\partial x_n} = 0$, (т.е. $\nabla L (x^0) = 0$) то эта точка - точка min выпуклой функции Лагранжа "L" в допустимой области. \\
Обозначим $min L(x) = L (x^0) = L^0 \Rightarrow L^0 \leqslant L$ \\
Но вследствие условия 2 теоремы Куна-Таккера \\
$\lambda_1 g_1^0 = ... = \lambda_m g_m^0 = 0 \Rightarrow L^0 = f^0, f^0 \leqslant L$ \\
Кроме того, в допустимой области $g_1 \leqslant 0, ..., g_m \leqslant 0$ и по условию 3 теоремы Куна-Таккера, $\lambda_1 \geqslant 0, ..., \lambda_m \leqslant 0 \Rightarrow \lambda_1 g_1 = ... = \lambda_m g_m \leqslant 0 \Rightarrow$ \\
в допустимой области \\
$L = f + \lambda_1 g_1 + ... + \lambda_m g_m \leqslant f \Rightarrow$ в допустимой области: \\
$f^0 \leqslant L \leqslant f$, т.е. $f^0 \leqslant f \Rightarrow min f = f^0 = f(x^0)$, т.е. точка $x^0 = (x_1^0, ..., x_n^0)$
- точка минимума (глобального) целевой функции в допустимой области. 
Т. О. достаточность теоремы Куна-Таккера доказана. \\
ч и т.д. 
\subsection*{Теорема Куна-Таккера (в "седловом" варианте)} \\
Точка $x^0 = (x_1^0, ..., x_n^0)$ является решением ЗВП (т.е. точкой глобального min-а функции $"f"$) $\Leftrightarrow \exists$ неотрицательный вектор множителей Лагранжа \\
$\lambda^0 = (\lambda_1^0, ... , \lambda_m^0) (\lambda_i^0 \geqslant 0; i = \overline{1, m})$ такой, что для функции Лагранжа \\
$L = (x_1, ..., x_n; \lambda_1, ..., \lambda_m) = f(x_1, ... , x_n) + \Sigma_{i = 1}^m \lambda_i g_i (x_1, ... , x_n)$ точка $(x^0; \lambda^0)$ является седловой точкой, т.е. \\
$L(x^0, \lambda) \leqslant (x^0, \lambda^0) \leqslant L(x, \lambda^0)$ (*) \\
$\forall x \in$ допустимой области, $\lambda \geqslant 0$ \\
(без доказательства) \\

\textbf{Rem:} \\
1. Из неравенства (*) следует, что точка - точка минимума функции, а точка - точка максимума функции по. Существование седловой точки означает равенство минимакса максимуму. \\
2. Можно показать, что необходимые и достаточные условия того, чтобы являлась седловой точкой функции Лагранжа, имеют вид: \\

\textbf{Rem: }
1. При доказательстве необходимости условий теоремы Куна-Таккера выпуклость функций и допустимой области не требовалась $\Rightarrow$ условия Куна-Таккера являются необходимыми для произвольной (необязательно выпуклой) ЗМП. \\
2. Множители Лагранжа $"\lambda_k"$ представляют собой "цену" ограничения $"g_k \leqslant 0"$, т.е. чувствительность оптимального значения целевой функции $"f^0"$ к нарушению этого ограничения. В функции Лагранжа слагаемые $"\lambda_k g_k"$ представляет собой как бы "штраф" за нарушение ограничения $"g_k \leqslant 0"$. Чем больше значение множителя Лагранжа $"\lambda_k"$ и чем больше нарушено /-щее ограничение (т.е. стало $"g_k > 0"$), тем больше величина этого "штрафа". На этой идее основан один из методов решения ЗМП - метод штрафных функций, который был рассмотрен ниже. \\

3. Условия теоремы Куна-Таккера представляют собой систему нелинейных (в общем случае) уравнений и неравенств, которая допускает лишь приближенное решение численными методами. Для решения ЗВП разработали специальные итерационные численные методы, которые заключаются в построении последовательных приближений к точке целевой функции в допустимой области. Критерием окончания итерационного процесса является достижение заданной точности вычислений. В результате определяются приближенные значения. \\

4. Точное решение ЗВП допускает в случае, когда и целевая функция и все ограничения являются линейными функциями переменных (линейные функции являются выпуклыми в /-ии со свойством 1 выпуклых функций). Такая ЗВП называется задачей линейного программирования (ЗЛП).

\subsection*{Линейное программирование}
\end{document}