\ProvidesFile{lecture08.tex}[Лекция 8]

\newpage

\section{Лекция №8. Определение первоначального допустимого базисного решеения. Особые случаи симплексного метода. Двойственные задачи линейного программирования.}
\subsection{Определение первоначального доспустимого базисного решения}
Базисные решения, получаемые на I шаге, не всегда являются допустимыми.\\
Рассмотрим один из алгоритмов получения допустимого базисного решения (ДБР):\\
\textbf{1.} Если в каждом уравнении дополнительная переменная и свободный член, стоящий в правой части, имеют одинаковые знаки, 
то дополнительные переменные берём в качестве базисных и при этом получаем ДБР.\\
\textbf{2.} Если хотя бы в одном уравнении дополняется переменная и свободный член 've противоположные знаки (и дополнительные  переменные в качестве базисных
то есть 1 базисное решение get-ся недопустимым), то в системе (1) (в которой базисные выражаются через свободные) рассм-ем $\forall$ ур-е с отрицательынм свободным членом и переводим в базисные $\forall$  из свободных переменных, входящих в это уравнение с положительным коэфицентом. Процедура повторяется до достижения ДБР.
При этом (при возможности выбора) следует переводить из свободных в базисные ту свободную переменную, которая def-т в качестве разрешающего уравнения с отрицательным свободным членом.
Только в этом случае новые базисные решения б. 've меньше отрицательныз компонент.\\
\textbf{3.} Если базисное решения недопустимое и в уравнениях с отрицательным свободным членом $\nexists$  свободных переменных с положительным коэффицентами, то $\nexists$ ДБР (то есть условия ЗЛП противоречивы). \\
\textbf{ex:}\\ 
$F=x_1+x_2 \rightarrow max$
при \eqn{
    \begin{cases}
    x_1-x_2 \leq-1 \\
    x_1-x_2 \geq-3 \\
    x_1 \leq3\\
    x_1, x_2 \geq0
    \end{cases}
}
В каноническом виде 
\eqn{
    \begin{cases}
        x_1-x_2+x_3 \ \ \ \ \ \ \ \ \ \ \ \ \ \ =-1 \\
        x_1-x_2 \ \ \ \ \ \ -x_4 \ \ \ \ \ \ \ \ = -3 \\
        x_1 \ \ \ \ \ \ \ \ \ \ \ \ \ \ \ \ \ \ \ \ \ + x_5 =3 \\
        x_i \geq0 \ (i=\overline{1,5})
    \end{cases}
}
\textbf{I шаг}\\
В соответствии с правилом со стр. ??? в качестве базисных берём дополнительные переменные $(x_3, x_4, x_5)$ ; свободные - $(x_1, x_2)$. Базисные - через свободные:
\eqn{
    \begin{cases}
        x_3 = -1-x_1+x_2 \\
        x_4 = 3+x_1-x_2\\
        x_5=3-x_1
    \end{cases}
} $\Rightarrow$ базисныое решение $X_1 = (0;0;-1;3;3)$ - не является допустимым, то есть оно $\notin$ многоугольнику решений $\Rightarrow \nexists F(X_1)$ \\
Уравнение с отрицательным свободным членом --- 1-ое, в нём с положительным коэфицентом --- $''x_2'' \Rightarrow $ её в базисные (так как пр и этом будет увеличиваться $''x_3''$) \\
Оценочные отношения $\Rightarrow$ Наибольшее возможное зачение: $x_2 = min (1;3; \infty ) \Rightarrow$ 1-ое уравнение - разрпешающее \\
\textbf{II шаг}\\
Базисные --- $(x_2; x_4; x_5)$ ; свободные --- $(x_1, x_3)$
\eqn{
    \begin{cases}
        x_2 = 1+x_1+x_3 \\
        x_4 = 3+x_1-(1+x_1+x_3)=2\\
        x_5=3-x_1
    \end{cases}
} 
$\Rightarrow$ get ДБР: $X_2=(0;1;0;2;3)$; \\
\eqn{
    F=x_1+(1+x_1+x_3)=1+2x_1+x_3;
}
\cent{
    $F(X_2)=1 \neq F_{max}$
}
$\Rightarrow$ следующий шаг симплексного метода...
\subsection{Взаимно двойственные задачи линейного программирования(ЛП), их свойстваю Основное неравенство теории двойственности. Первая(основная) теорма двойственности.}
\subsubsection{Взаимно двойственные задачи ЛП и их свойства}
$\forall$ ЗЛП соответствует двойственная (сопряжённая) ей задача ЗЛП.\\
\textbf{ex:} (см стр. ??? об use-нии ресурсов):\\
Из ресурсов $S_i$ (i= $\overline{1,m}$) с запасами $b_i (i= \overline{1,m})$ изготваливаются виды продукции $P_j (j= \overline{1,n})$ в количестве $x_j (j= \overline{1,n})$. Технологические коэффиценты --- $a_{ij} (i= \overline{1,m}, j= \overline{1,n})$ ---
число единиц ресурса $s_i$ затртивших на изготовление одной единицы продукции $p_j$ прибыль(выручка) от реализации единицы продукции $p_j$ --- $c_j, j= \overline{1,n}$ (то есть цена единицы продукции $p_j$)\\
Целевая фунция: $F = \Sigma_{j=1}^n c_j*x_j \rightarrow max $ --- прибыль от реализации всей продукции. \\
При ограничениях: $\Sigma_{j=1}^n a_{ij}*x_j \leq b_i$, ($i=\overline{1,m}$)\\
Пусть другая организация хочет купить у этого предприятия все ресурсы $S_i$ $i=\overline{1,m}$.\\
Целевая функция: $Z = \Sigma_{i=1}^m b_i*y_i \rightarrow min$ --- затраты на покупку всех ресурсов по ценам $''y_i''$ \\
При ограничениях $Z = \Sigma_{i=1}^m a_{ij}*y_i \geq c_j, (j=\overline{1,n})$ --- выручка продавца (предприятия) должна быть не менее той суммы, которую предприятие может получить при переработке ресурсов $S_i$ в готовую продукцию $P_j$.\\
\textbf{Экономико-математическая модель исходной и двойственной задач:} \\
Задача I (исходная):
\eqn{
    F=c_1*x_1+...+c_nx_n \rightarrow max (1)
}
\eqn{
    \begin{cases}
    a_{11}*x_1+a_{12}x_2+...+a_{1n}x_n \leq b_1\\
    \ \ \ \ \ \ \ \ \ \ \ \ \ \ \ \ \ \  \ \  \ \ \dots \ \ \ \ \ \ \ \ \ \  \ \ \ \ \ \ \ \ \ \ \ \ \ (2)\\
    a_{m1}*x_1+a_{m2}x_2+...+a_{mn}x_n \leq b_m
    \end{cases}
}
\eqn{x_j \geq0 \ (i=\overline{1,n}) \ \ \ (3)}
Найти: план выпуска продукции 
