\ProvidesFile{lecture06.tex}[Лекция 6]

\newpage

\section{Лекция №8. Определеение первоначального допустимого базисного решеения. Особые случаи симплексного метода.}
\subsection{Определение первоначального доспустимого базисного решения}
Базисные решения, получаемые на I шаге, не всегда являются допустимыми.\\
Рассмотрим один из алгоритмов получения допустимого базисного решения (ДБР):\\
\textbf{1.} Если в каждом уравнении дополнительная переменная и свободный член, стоящий в правой части, имеют одинаковые знаки, 
то дополнительные переменные берём в качестве базисных и при этом получаем ДБР.\\
\textbf{2.} Если хотя бы в одном уравнении дополняется переменная и свободный член 've противоположные знаки (и дополнительные  переменные в качестве базисных
то есть 1 базисное решение get-ся недопустимым), то в системе (1) (в которой базисные выражаются через свободные) рассм-ем $\forall$ ур-е с отрицательынм свободным членом и переводим в базисные $\forall$  из свободных переменных, входящих в это уравнение с положительным коэфицентом. Процедура повторяется до достижения ДБР.
При этом (при возможности выбора) следует переводить из свободных в базисные ту свободную переменную, которая def-т в качестве разрешающего уравнения с отрицательным свободным членом.
Только в этом случае новые базисные решения б. 've меньше отрицательныз компонент.\\
\textbf{3.} Если базисное решения недопустимое и в уравнениях с отрицательным свободным членом $\nexists$  свободных переменных с положительным коэффицентами, то $\nexists$ ДБР (то есть условия ЗЛП противоречивы). \\
\textbf{ex:}\\ 
$F=x_1+x_2 \rightarrow max$
при \eqn{
    \begin{cases}
    x_1-x_2 \leq-1 \\
    x_1-x_2 \geq-3 \\
    x_1 \leq3\\
    x_1, x_2 \geq0
    \end{cases}
}
В каноническом виде 
\eqn{
    \begin{cases}
        x_1-x_2+x_3 \ \ \ \ \ \ \ \ \ \ \ \ \ \ =-1 \\
        x_1-x_2 \ \ \ \ \ \ -x_4 \ \ \ \ \ \ \ \ = -3 \\
        x_1 \ \ \ \ \ \ \ \ \ \ \ \ \ \ \ \ \ \ \ \ \ + x_5 =3 \\
        x_i \geq0 \ (i=\overline{1,5})
    \end{cases}
}
\textbf{I шаг}\\
В соответствии с правилом со стр. ??? в качестве базисных берём дополнительные переменные $(x_3, x_4, x_5)$ ; свободные - $(x_1, x_2)$. Базисные - через свободные:
\eqn{
    \begin{cases}
        x_3 = -1-x_1+x_2 \\
        x_4 = 3+x_1-x_2\\
        x_5=3-x_1
    \end{cases}
} $\Rightarrow$ базисныое решение $X_1 = (0;0;-1;3;3)$ - не является допустимым, то есть оно $\notin$ многоугольнику решений $\Rightarrow \nexists F(X_1)$ \\
Уравнение с отрицательным свободным членом --- 1-ое, в нём с положительным коэфицентом --- $''x_2'' \Rightarrow $ её в базисные (так как пр и этом будет увеличиваться $''x_3''$) \\
Оценочные отношения $\Rightarrow$ Наибольшее возможное зачение: $x_2 = min (1;3; \infty ) \Rightarrow$ 1-ое уравнение - разрпешающее \\
\textbf{II шаг}\\
Базисные --- $(x_2; x_4; x_5)$ ; свободные --- $(x_1, x_3)$


