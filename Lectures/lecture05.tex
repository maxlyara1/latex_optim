\ProvidesFile{lecture05.tex}[Лекция 5]

\newpage

\section*{Лекция №5. Общая постановка выпуклого программирования(ЗВП).Теорема Куна-Таккера}
\subsection*{Общая постановка ЗВП}
ЗВП называется следующей задачей математического программирования(ЗМП)\\
$f(x_1,...,x_n) \rightarrow min$\\
при ограничениях:
\eqn{
\begin{cases}
g_1(x_1,...,x_n) \leq 0 \\
g_m(x_1,...,x_n) \leq 0
\end{cases},}
где $f,g_1,...,g_m$ - выпуклые функции, def-ные на некотором выпуклом множестве $M\subset R^n$.\\
При этом множество $M$ содержит допустимую область значений, то есть множество, удовлетворяющее системе ограничений. Заметим, что свойству 3.1 выпуклых функций допустимая область ЗВП также выпукла \\
\textbf{Rem:}\\
1) Аналогично формулируется задача максимизации вогнутой функции $"f"$ при вогнутых функциях $"g_1,...,g_m"$, def-х на некотором выпуклом множестве $M\subset R^n$. При этом знак неравенств --- $"\geq"$.\\
Аналогично допустимая область ЗВП будет также выпуклым множеством(свойство для вогнутых функций)\\
2) В ??? с теор. на с.??? в ЗВП каждый локальный минимум является глобальным минимумом функции $"f"$ в допустимой области. Причём, если функция $"f"$ строго выпуклая и ограниченная снизу на ограниченном непустом множестве $M$, то ЗВП 've единственное решение, то есть минимумом функции $"f"$ достигается в одной точке
\eqn{x^0=(x^0_1,...,x_n^0)} (см с.???)

\subsection*{Теорема Куна-Таккера.}
\textbf{Теорема:} Пусть на ЗВП налагаются следующие требования:\\
1. Множество, удовлетворяющее системе строгих неравенств:
\eqn{\begin{cases} g_1(x_1,...,x_n)<0 \\ g_m(x_1,...,x_n)<0\end{cases}}
и называемое внутренней частью допустимой области, не пусто (т.н. условие Слейтера) \\
2. Часть допустимой области, в которой некоторые ограничения обращаются в равенства, называется границей допустимой области, а эти ограничения - активными. Пусть градиенты активных ограничений в отвечающих им точках границы линейно независимы(т.н. условие регулярности (см с.???))\\

\subsection*{Теорема Куна-Таккера (необходимые и достаточные условия решения (глобального минимума) ЗВП)} 
Точка $x^0=(x_1^0,...,x_n^0)$ является решением ЗВП (то есть точкой глобального минимума функции $"f"$) $\Leftrightarrow$(Н. и Д.) в ней выполнены следующие условия(Условия Куна-Таккера):\\
1)Условие стационарности функции Лагранжа (см.с.???)
\eqn{\lambda (x_1,...,x_n; \lambda_1,..., \lambda_m)=f(x_1,...x_n)\Sigma_{i=1}^m)*\lambda_i*g_i*(x_1,...,x_n)}
\cent{$\frac{\partial L}{\partial x_1} =...= \frac{\partial L}{\partial x_n} = 0$ (то есть  $\triangledown(x^0)=0$);}
2)Условия "дополняющей нежёсткости":
\eqn{\lambda_i * g_i*(x_1^0,..., x_m^0)=0 \ (i= \overline{1,m});}
3)Условия неотрицательности:
\eqn{\lambda_i \geq 0 \ (i=\overline{1,m})}
\textbf{Доказательство:}\\
Заметим, что в условиях К.-Т. множители Лагранжа $\lambda_1,..., \lambda_m$ являются "выключателями" делящих ограничений. Если, например, $\lambda_k=0$, то ограничение $g_k \leq 0$ исключается из условий К.-Т., так как : 1) это ограничение не входит в функцию Лагранжа (слагаемое $\lambda_k*g_k = 0$) и,??? не входит в условие 1; 2) условия 2 и 3 для этого ограничения выполняются автоматически\\
\textbf{1. Необходимость}\\
Пусть точка $x^0=(x_1^0,...,x_n^0)$ - решение ЗВП, то есть $minf(x_1,...,x_n)=f(x_1^0,...x_n^0)$. Целевая функция $"f"$ может достигать минимума внутри допустимой области или на её границе.\\
Пусть минимум достигается внутри области $\Rightarrow$ в точке $x^0$ выполняются необходимые условия локального(безусловного экстремума функции $"f"$):
\eqn{\frac{\partial f}{\partial x_1}=...=\frac{\partial f}{\partial x_n}}
Это частный случай условия 1 К.-Т.(условия стационарности функции Лагранжа) при $\lambda_1=...=\lambda_m = 0$ все ограничения выключены, так как 've безусловную оптимизацию)\\
Условия 2 и 3 при этом (при $\lambda_i = 0 (i=\overline{1,m}$) выполняются автоматически.\\
Пусть минимум достигается на границе допустимой области, то есть является условным extr функции $"f"$ с активными ограничениями задачи для точки минимума точки $x^0$. Необходимые условия условного extr (см. с. ???) $(\frac{\partial L}{\partial x_j}=0;(\frac{\partial L}{\partial \lambda_j}=0 (j=\overline{1,n}; i=\overline{1,m}))$ влекут за собой выполнение условий 1 и 2 К.-Т., если положительно равынми нулю множит. Лагранжа для неактивных ограничений (так как реально они не входят в функцию Лагранжа). Остаётся доказать, что выполняется 3 условие К.-Т. :$\ \lambda_i \geq 0 (i=\overline{1,m})$ \\
Для наглядности ограничимся случаем двух переменных (n=2).\\
Уравнение $g(x_1,x_2)$ означает плоскую кривую, а неравенство $g(x_1,x_2) \leq  0$ - одну из частей плоскости, ограниченную этой кривой\\
Система ограничений \eqn{\cs{g_1(x_1,x2)\leq 0\\
...\\
g_m(x_1,x_2) \leq 0
}}
задает криволинейный многоугольник, в каждой вершине которого пересекаются две кривые.\\
Минимум на границе допустимой области достигается либо на строке этого многоугольника, либо в его вершине.\\
Пусть минимум - на стороне многоугольника \Rightarrow активным является только одно ограничение и только отвечающий ему множитель Лагранжа $\lambda_k \neq 0$. Функция Лагранжа 've вид:
\eqn{L=f(x_1,...x_n)+\lambda_k*g_k(x_1,...,x_n),} где $g_k=0$ - активное ограничение.\\
В точке минимума выполняются необходимые условия.
Условия условного extr(см. с. ???):
\eqn{\frac{\partial L}{\partial x_1}=...\frac{\partial L}{\partial x_n}=\frac{\partial L}{\partial \lambda_k}} или 
\eqn{\cs{\triangledown L = \triangledown f + \lambda_k * \triangledown g_k=0\\
g_k = 0},}
$\Rightarrow \triangledown f = -\lambda_k * \triangledown g_k , (*)$, \\
то есть векторы $\triangledown f$ и $\triangledown g_k$ --- коллинеарны\\
Функция $g_k < 0$ внутри допустимой области и $g_k > 0$ вне её $\Rightarrow$ вектор $\triangledown g_k$ направлен из допустимой области (в сторону возрастания функции $"g_k"$)\\
Значение функции $"f"$ внутри допустимой области больше, чем в точке минимума $\Rightarrow \triangledown f$ направлен внутрь допустимой области. \\
Таким образом, векторы $\triangledown f$ и $\triangledown g_k$ противонаправлены и в равенстве (*) коэффиценты коллинеарности $\lambda_k \leq 0 \Rightarrow \lambda_k \geq 0 $, то есть условие 3 К.-Т. доказано для случая, когда минимум достигается на стороне допустимого многоугольника.\\
Можно показать, что условие 3 К.-Т. $(\lambda_i \geq 0 (i=\overline{1,m}))$ выполняется и для случая, когда минимум достигается в вершине допустимого многоугольинка.\\
Таким образом, необходимость К.-Т. доказана.

