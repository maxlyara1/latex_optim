\ProvidesFile{lecture01.tex}[Лекция 1]

\newpage

\section{Лекция 1. Безусловная оптимизация}
%\section*{Безусловная оптимизация}
\subsection*{Постановка задачи и определения}

\textbf{def:} Методы оптимизации --- это раздел математики, посвящённый решению (экстремальных) задач, то есть задач на нахождение минимумов и максимумов функций.\\ [2mm]
\textbf{Rem:} Задачу на нахождение максимума функции "f(x)" можно свести к задаче на нахождение минимума функции $"f_1(x) = -f(x)"$ и наоборот.
\subsection*{Общий вид оптимизационной задачи}
Найти экстремум(минимум или максимум) функции $f: X \rightarrow R$ определенной на некотором множестве $X \in R^n$ при ограничении $X \in D (D \subset X)$ то есть $f(x) \rightarrow extr, X\in D$(у функции есть экстремум на промежутке D).\\
В большинстве задач область определения функции $"f(x)" \  X=R^n$ . Ограничение $X\in D$ записывается, как правило, в виде уравнений или неравенств . Если множество $D=X$, то имеет место задача без ограничений или {задача безусловной оптимизации}. \\
При решении оптимизационной задачи находятся как {локальные}, так и {глобальные экстремумы функции}. \\[2mm]
\textbf{def:} Точка $" x^* "$ является точкой {локального минимума(максимума) функции} $"f(x)" $ if $ \exists$ $"\upvarepsilon "$ - окрестность $ \mathcal{U}_ \upvarepsilon = \{ x :  |x-x^*| \} < \upvarepsilon $ .
Точка $x^* : f(x^*) \leq f(x) \  (f(x^*) \geq f(x)) \ \forall x \in \mathcal{U}_ \upvarepsilon $ \\ [2mm]
\textbf{Rem:} то что пишется в скобках - для максимума, а то что без - для минимума \\ [2mm]
\textbf{def:} Точка $"x^*" $ является точкой {глобального минимума(максимума)} функции $"f(x)"$ if $ f(x^*) \leq f(x) \  (f(x^*) \geq f(x)) \  \forall x \in D $

\subsection*{Необходимые и достаточные условия экстремума}
\textbf{def:} Точка $X^0=(x^0_1,...,x^0_n)$ называется стационарной точкой дифференцируемой функции $f(x)=f(x_1... x_n)$, если в ней все частные производные равны нулю, то есть $f'(x^0)=0$ или $\frac{\partial f(x^0)}{\partial x_1} = ... = \frac{\partial f(x^0)}{\partial x_n} = 0$
\subsection*{Необходимые условия экстремума I порядка}
\textbf{Теорема: }if точка $x^*=(x_1^*, ..., x_n^*)$ - точка локального extr дифференцируемой в точке $x^* $ функции $ f(x_1, ... x_n) \Rightarrow then \ \frac{\partial f(x^*)}{\partial x_1} = ... = \frac{\partial f(x^*)}{\partial x_n} = 0$ $(1)$ (то есть - точка $x^{*}$ - точка локального экстремума $\Rightarrow$ точка $x^{*}$ - стационарная точка (обратное утверждение неверно)) \\ [2mm]
\textbf{Доказательство:} Рассмотрим функцию одной переменной:\\ $\upvarphi (x_i)=f(x_1^*,..., x_{i-1}^*, x_i^*, x_{i+1}^*, x_n^*)$ точка $x^*=(x_1^*,...,x_n^*)$ - т. локального extr функции $"f" \Rightarrow x_i^*$ - т. локального extr функции $"\upvarphi" \Rightarrow$ по необходимому условию для функции одной переменной (по т. Ферма) 've:
\begin{center}
$\upvarphi(x_i^*)=0 \Leftrightarrow $ $\frac{\partial f(x^*)}{\partial x_i} = 0$ --- что и требовалось доказать :)
\end{center}
Для формулировки достаточных условий extr, позволяющих отобрать среди стационарных точек именно точки локального extr(среди стационарных точек могут быть также точки перегиба, седловые точки и т.д.), рассмотрим матрицу вторых производных функции - матрицу Гессе(гессиан):
\cent{$A=f^{''}(x^*) = (\frac{\partial ^2 f(x^*)}{\partial x_i \partial x_j})_{i, j = \overline{1, n}} = (a_{i, j})_{i, j = \overline{1,n}}$ (от 1 до n)}
\textbf{def:} Матрица "A" называется неотрицательно определённой $(A \geq 0),$  если $\forall h = (h_1,...,h_n) \in R^n$ неотрицательной является квадратичная форма:
\cent{$(A*h, h) = \Upsigma_{i, j = 1}^n a_{ij}h_ih_j \geq 0$}
\textbf{def:} Матрица "A" называется положительно определённой $(A>0)$, если $(A*h, h)>0, \forall h \in R^n (h \neq 0)$

\subsection*{Необходимые и достаточные условия extr II порядка}
\textbf{Теорема:} Пусть $f(x)$ - дважды дифференцируема в точке $x^*$. Необходимые условия условия extr: \\
if точка $x^*$ - точка локального минимума (максимума) функции $
f(x) \Rightarrow \  f^{'}(x^*)=0; (f^{''}(x^*)*h, h) \geq 0 \ ((f^{''}(x^*)*h, h) \leq 0) \forall h \in R^n$
\subsection*{Достаточные условия extr} $f^{'}(x*)=0; (f^{''}(x^*)*h, h)$ \textgreater $0 \ ((f^{''}(x^*)*h, h)$ \textless $0) \forall h \in R^n (h \neq 0)$ $\Rightarrow $ точка $x^*$ - т. локального минимума (максимума) функции $f(x)$\\
\textbf{Доказательство:} \\
Для случая минимума (для максимума аналогично) \\
По формуле Тейлора 've:
\cent{$f(x^*+h) = f(x^*) + (f^{'} (x^*), h) + \frac{1}{2}(f^{''}(x^*)*h, h) + r(h)$,\\ где $r(h)=o(|h|^2). (*)$}
\textbf{1) Необходимость:} \\
Пусть точка $x^*$ - точка локального минимума $\Rightarrow$ по необходимому условию I порядка $f^{'}(x^*)=0$, а также $f(x^* + \uplambda h) \geq f(x^*)$ (при достаточно малых $"\uplambda"$) $\Rightarrow$ из (*) get (g при малых $"\uplambda"$ и фиксированном $"h"$):
\cent{$f(x^{*}+\uplambda h) - f(x^*) = 0 + \frac{\uplambda^2}{2} (f^{''}(x^*)*h, h) + r( \uplambda*h) \geq 0 |: \uplambda^2$\\
(где $r( \uplambda*h) =0 (|\uplambda|^2)).$ 
$\frac{1}{2}(f''(x^*)*h, h)+\frac{r(\lambda *h)}{\lambda^2} \geq 0$}\\
При $\lambda \rightarrow 0 \ 've: (f''(x^*)*h)\geq 0 (\forall h \in R^n) \Rightarrow $ необходимость доказана \\
\textbf{2) Достаточность:}\\
Можно показать, что в $R^n$ имеет место эквивалентность условий:
\eqn{(A*h, h)>0 \forall h \in R^n (h \neq 0) \Leftrightarrow \exists \alpha > 0: (A*h, h) \geq \alpha*|R|^2 \ (\forall h \in R^n)}
Учитывая, что $f'(x^*)=0$ и $(f''(x^*)*h,h)\geq \alpha * |h|^2$ \\
По формуле Тейлора 've:\\
$f(x^*+h)-f(x^*)=0+\frac{1}{2}(f''(x^*)*h,h)+r(h)\geq \frac{\alpha}{2}|h|^2+r(h) \geq 0))$, то есть $f(x^*+h) \geq f(x^*) \Rightarrow$ точка $x^*$ - точка локального extr функции $f(x) \Rightarrow$ достаточность доказана.
\begin{center}
что и требовалось доказать
\end{center}
\textbf{Rem:} 
Для квадратичной функции\\
$Q(x)=\Sigma^n_{i,j=1} a_{ij}x_i x_j$ условие положительной(отрицательной) определённости матрицы $A=(a_{ij})^n_{i,j=1}>0$ - это достаточное условие абсолютного минимума(максимума) $Q(x)$ в стационарной точке.

\subsection*{Теорема Вейерштрасса}
\textbf{Теорема:} \\
Непрерывная функция $f:R^n \rightarrow R$ на непустом ограниченном замкнутом подмножестве(компакте) множества $R^n$ достигает своих абсолютных минимума и максимума[или 1)в стационарной точке внутри; 2) в граничной точке] - без доказательства\\
\textbf{Следствие:} \\
if функция $"f(x)"$ непрерывна на $R^n$ и $\displaystyle{\lim_{|x| \to \infty}{f(x)}= \infty } $ \\
$\displaystyle{(\lim_{|x| \to \infty}{f(x)} = -\infty )} $ 
$\Rightarrow$ then она достигается своего абсолютного минимума (максимума) на $\forall$ замкнутом подмножестве и $R^n$. (без доказательства).

\subsection*{Критерий Сильвестра}
\textbf{Rem:} В необходимых и достаточных условиях extr-а II порядка use-ся знакоопределённость матрицы вторых производных (гессиана) $A=f''(x)$.\\
Знакоопределённость матрицы устанавливается с помощью критерия Сильвестра.\\
\textbf{Теорема:}\\
Пусть A - симметричная матрица\\
1) Матрица "A" положительно определена $(A>0)$ $\Leftrightarrow$ все её последовательные гл. миноры положительны, т.е. $A_{1...k} = \begin{vmatrix}
a_{11} & a_{ik} \\
a_{k1} & a_{kk}\\
\end{vmatrix} >0 \ (k = \overline{1,n})
$ \\
2) Матрица "A" отрицательно определена $(A<0)$ $\Leftrightarrow$ все её последовательные главные миноры чередуют знак, начиная с отрицательного, т.е. $(-1)^k*A_k >0 $ \ (k = $\overline{1,n} )$ \\
3) Матрица "A" неотрицательно определена $(A\geq 0) \Leftrightarrow$ все её гл. миноры (необязательно только последовательные) неотрицательны, т.е. $A_{1...k} = \begin{vmatrix}
a_{i_1 i_1} & a_{i_1 i_k} \\
a_{i_k i_1} & a_{i_k i_k}\\
\end{vmatrix} \geq 0 \ (1 \geq i_1 \geq ... \geq i_k \geq n) (k = \overline{1,n})
$ \\
4) Матрица "A" неположительно определена $(A\leq 0)$ $\Leftrightarrow$ все её последовательные главные миноры чередуют знак, начиная с неположительного, т.е. $(-1)^k*A_{i_1...i_k} \geq 0 $ \ (k = $\overline{1,n} )$ \\ 
(Теорема без доказательства)\\
\textbf{Rem:} \\
1) Можно показать, что $A>0(A \geq 0) \Leftrightarrow \forall \lambda_i >0 (\lambda \geq 0)$, где $\lambda_i$ - собственные значения матрицы. \\
2)\\
2.1) $A_{1...k} > 0 \Leftrightarrow A_{i_1...i_k} > 0$ \\
2.2) $A_{1...k} \geq 0 \nRightarrow A_{i_1...i_k} \geq 0$ (т.е. $\nRightarrow A \geq 0$) \\
\textbf{ex:} \\
$
A = \begin{pmatrix}
0 & 0 \\
0 & -1 \\
\end{pmatrix} 
$
$\Rightarrow$ последовательные главные миноры $A_1=0; A_{12}=
\begin{vmatrix}
0 & 0 \\
0 & -1\\
\end{vmatrix}=0
$, но A не является неотрицательно определённой, так как $(Ah, h)=((0;-h),(h,h)) = -h^2<0 (\forall h \neq 0)$

\subsection*{Правило решения задачи безусловной оптимизации}
1) Найти стационарные точки, то есть точки, удовлетворяющие необх. усл. extr I порядка \\
$$
\begin{cases}
\frac{\partial f}{\partial x_1} = 0 \\
\frac{\partial f}{\partial x_n} = 0
\end{cases}
$$
\\
2) Во всех стационарных точках $"x^0"$ проверяем достаточное условие extr II порядка, то есть проверяем знаки последовательных главных миноров гессиана $"f^{''}(x^0)":$ \\
2.1) if $A_{1..k}>0$ (k от 1 до n) $\Rightarrow then \ x^0 \in locminf;$ \\
2.2) if $(-1)^k*A_{1...k} > 0$ (при k от 1 до n) $\Rightarrow then \ x^0 \in locmaxf;$ \\
3) Если достаточное условие extr II порядка не выполняется $\Rightarrow$, то проверяем в стационарной точке необходимое условие extr II порядка, то есть проверяем знаки главных миноров гессиана $"f^{''}(x^0)"$ \\
3.1) if гессиан $f^{''}(x^0) \ngeq 0$ не является неотрицательным отрезком, то есть не выполняется условие $A_{i1...ik} \geq 0 \Rightarrow then \ x^0 \notin locminf;$ \\
3.2) if гессиан $f^{''}(x^0) \nleq 0$ не является неположительным отрезком, то есть не выполняется условие что все $(-1)^{k}A_{i1...ik} \geq 0 \Rightarrow then \ x^0 \notin locmaxf;$ \\

