\ProvidesFile{lecture03.tex}[Лекция 3]

\newpage

\section{Лекция №3. Выпуклый анализ}
Ввиду важности задачи выпуклого программирования (ЗВП) рассматриваемые основные понятия раздела математики, называемого выпуклым анализом: \\ 1) Выпуклое множество 2) Выпуклая(вогнутая) функция и ее дифференциальные свойства

\subsection{Выпуклые множества точек}
\textbf{def:}
Множество точек называется выпуклым, если оно вместе с $\forall$ двумя своими точками содержит весь отрезок, соединяющий эти точки
\textbf{ex:} --- Тут нужно иллюстрацию
\textbf{ex:} круг, сектор, отрезок, прямая, полуплоскость, куб, пирамида

\begin{theorem}
Пересечение $\forall$ числа выпуклых множеств-выпуклое множеств, too
\end{theorem}

\renewcommand\qedsymbol{$\blacksquare$}

\begin{proof}
Для простоты - пересечение двух выпуклых множеств: \\ $\forall N,M \in (A \cap B)$. Множество А - выпуклое $\Rightarrow$ отрезок $MN \in A$. Аналогично \\ $MN \in B $$\Rightarrow$$ MN \in (A \cap B) $$\Rightarrow$$ (A \cap B)$ - выпуклое --- тут нужно иллюстрацию
\end{proof}

\textbf{def:} Внутренняя т. множества - $\exists$ окрестность этой точки: в ней - только точек $\in$ множеству. \\

\textbf{def:} Граничная точка множества - $\forall$ окрестность этой точки содержит как точки $\in$ множеству, так и точек $\notin$ множеству \\

\textbf{def:} Угловая точка множества - она не является внутренней ни для какого отрезка, целиком $\in$-го множеству. \\

\textbf{ex:}
--- Тут нужно иллюстрацию
точка M - внутренняя, точка N граничная, точка A - угловая ($AP \in$ множеству целиком, но точка A - не внутренняя точка для AP; точка внутренняя точка для KL, но KL $\notin$ множеству целиком \\

\textbf{def:} Замкнутое множество точек - if оно включает все свои граничные точки. \\

\textbf{def:} Ограниченное множество точек - if $\exists$ шар конечного радиуса с центром в $\forall$ точке множества, который полностью содержит в себе данное множество.  \\

Можно показать, что если фигура ограничена only прямыми или их отрезками, то она 've конечное число угловых точек. При криволинейности границ - бесконечное множество угловых точек.

\textbf{def:} Выпуклое замкнутое множество точек пространства(плоскости), имеющее конечное число угловых точек, называется выпуклым многогранником(многоугольником), если оно ограниченное, и выпуклой многогранной(многоугольной) областью, если оно неограниченное. \\



\textbf{Rem:} Для выпуклого многогранника(многоугольника) угловые точки $\equiv$ его вершинам. \\
Для невыпуклого - не обязательно

\textbf{ex:} --- Тут нужна иллюстрация точка E - вершина, но не угловая точка, т.к. KL $\in$ множеству целиком и точка E - внутренняя точка для KL.

\textbf{Rem:} В ЗЛП часто число параметров объекта $n > 3 \Rightarrow$ имеем дело с гипермногогранниками в гиперпространстве с координатами $x_i (i = (\overline{1,n}, n > 3)$

\subsection{Геометрический смысл решений СЛН и СЛУ}

\begin{theorem}
Множество всех решений линейного неравенства 
    \begin{equation*}
    a_{11} x_1 + ... + a_{1n}x_n \leq b_1
    \end{equation*} - это одно из полупространств, на которые n-мерные гиперпространство делится гиплоскостью
    
    \begin{equation*}
    a_{11} x_1 + ... + a_{1n}x_n = b_1
    \end{equation*} включая и эту гиперплоскость   
\end{theorem}

\section{Выпуклые множества в n-мерном пространстве. Свойства ЗЛП}
\subsection{Выпуклые множества в n-мерном пространстве}
Рассмотрим в n-мерном пространcтве k точек (векторов): \\
$x_1 = (x_1^{(1)},...,x_n^{(1)})$, ..., $x_k = (x_1^{(k)},...,x_n^{(k)})$

\textbf{def:} Точка(вектор) $X = (x_1,...,x_n)$ называется линейной комбинацией точек(векторов) $X_1,...,X_k$, если справедливо соотношение:

\begin{equation}
X = \alpha_1 X_1 + ... + \alpha_k X_k
\end{equation} где $\alpha_j$ - const $j = (\overline{1,k})$ \\

\textbf{def:} Точка(вектор) X называется выпуклой линейной комбинацией точек(векторов) $X = \alpha_1 X_1 + ... + \alpha_k X_k$, если:

\begin{enumerate}
\item \begin{equation*}X = \alpha_1 X_1 + ... + \alpha_k X_k \end{equation*}
\item \begin{equation}\alpha_j \geq 0 j = (\overline{1,k}) \end{equation}
\item \begin{equation}\sum_{j=1}^{k}\alpha_j = 1 \end{equation} 
\end{enumerate}



Очевидно, что в частном случае при k=2 выпуклой линейной комбинацией двух точек $X_1$ и $X_2$ является соединяющий их отрезок, т.к. 
\[ 
X = \alpha_1 X_1 + \alpha_2 X_2 =  
    \left\{
        \begin{array}{lr} 
        \alpha_1 + \alpha_2 = 1 & \\ 
        \alpha_2 = \alpha_1 - 1
        \end{array}
    \right\}
\alpha X_1 + (1-\alpha_1)X_2
\]

- уравнение точек $X \in [X_1, X_2]$ (см. аналитич. геометр.)
--- добавить иллюстрацию

