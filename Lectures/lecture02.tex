\ProvidesFile{lecture02.tex}[Лекция 2]

\newpage

\section{Лекция №2. Условная оптимизация}
\subsection{Условная оптимизация c ограничениями-равенствами. Функция Лагранжа}

\textbf{def:} Задачей условной оптимизации c ограничениями-равенствами называется следующая задача: \\ 
$f_1 \rightarrow extr$; $f_i(x)=0(i = \overline{1,m}),(m<n)$, где $f_i(x): R^n \rightarrow R(i = \overline{0,m})$ и $\forall f_i(x)$ - дифференцируема. \\

\begin{theorem}
необходимое условие экстремума I порядка.
\end{theorem}

If т.$x^* = (x^*_1,...,x^*_n)$ $\in locextr f_0 \Rightarrow$ then $\exists$ вектор множителей Лагранжа
$\lambda^* = (\lambda^*_1,...,\lambda^*_m)$ $\in R^m$ $(\lambda)$ $(\lambda^*\neq0)$ такой, что для функции Лагранжа:

\begin{equation}
\mathcal{L}(x_1,...,x_n;\lambda_1,...,\lambda_m) = f_0(x1,...,x_n) + \sum_{i=1}^{m} \lambda_i f_i(x_1,...,x_n)
\end{equation}

выполняется условия стационарности:

\begin{equation}
\begin{aligned}
\frac{\partial \mathcal{L}(x^*,\lambda^*)}{\partial x_j} &= \frac{\partial f_1(x^*)}{\partial x_j} + \sum_{i=1}^{m}\lambda_i \frac{\partial f_i(x^*)}{\partial x_j} = 0 \quad (j = \overline{1,n})
\\
\frac{\partial \mathcal{L}(x^*, \lambda^*)}{\partial \lambda_i} &= f_i(x^*) = 0 \quad (i = \overline{1,m})
\end{aligned}
\end{equation}

т.е. 've систему $n+m$ уравнений для нахождения $n+m$ неизвестных $\{x_1^*,...,x_n^*;\lambda_1^*,...,\lambda_m^*\}$ \\

\begin{theorem}
необходимое условия экстремума II порядка.
\end{theorem}
If т.$x^* = (x_1^*,...,x_n^*) \in locmin f_0$ (условие регулярности) и векторы $f'_1(x^*),...,f'_m(x^*)$ - линейно независимы $\Rightarrow$ then $\exists$ вектор множителей Лагранжа \\
$\lambda^* = (\lambda_1^*,..., \lambda_m^*) \in R^m (\lambda^* \neq 0)$ такой, что для функции Лагранжа 

\begin{equation*}
\begin{aligned}
\mathcal{L}(x,\lambda)=f_0(x) + \sum_{i=1}^{m}\lambda_i f_i(x)
\end{aligned}
\end{equation*}

выполняются условия:
\begin{enumerate}
\item стационарности (2)
\item неотрицательной определенности матрицы вторых производных: \\ 
(3) $(\mathcal{L}''(x^*,\lambda^*)h,h) \geq 0$ $\forall h \in \{ (f'_i(x^*),h) = 0 (i = \overline{1,m})\}$
\end{enumerate}
\textbf{Rem:} Для т.$x^* \in locmax f_0$ need $\mathcal{L}(x,\lambda)=-f_0(x) + \sum _{i=1}^{m} \lambda_i f_i(x)$ \\



\begin{theorem}
достаточное условие экстремума II порядка.
\end{theorem}
\noindent
If в т. $x^* = (x_1^*,...,x_n^*)$ векторы $f'_1(x^*),...,f'_m(x^*)$ - линейно независимы и $\exists$ вектор множителей Лагранжа $\lambda^* = (\lambda_1^*,...,\lambda_m^*) \in R^m (\lambda^* \neq 0)$ такой, что для функции Лагранжа 

\begin{equation*}
\begin{aligned}
\mathcal{L}(x,\lambda)=f_0(x) + \sum_{i=1}^{m}\lambda_i f_i(x)
\end{aligned}
\end{equation*}
 в т.$x^*$ выполняются условия:

\begin{enumerate}
\item стационарности (2)
\item положительной определенности матрицы вторых производных: \\
$(\mathcal{L}(x^*,\lambda^*)h,h)>0$ $\forall h \in \{(f'_i(x^*),h)=0 (i=\overline{1,m}),(h \neq 0)\}$ $\Rightarrow$ \\ $\Rightarrow$ then т. $x^* \in locmin f_0$ 
\end{enumerate}
\textbf{Rem:} Для т.$x^* \in locmax f_0$ need $ \mathcal{L}(x,\lambda) = -f_0(x) + \sum_{i=1}^{m}\lambda_i f_i$ \\ 

Частный случай \\
$z= f(x,y) \rightarrow$ extr; при $\varphi(x,y) = 0$ $\Rightarrow$ функция Лагранжа 've вид: \\
$\mathcal{L}(x,y,\lambda) = f(x,y) + \lambda \varphi(x,y)$ \\

Условия стационарности(необх. усл. I порядка)(2): \\
\eqn{\begin{cases}
\mathcal{L}'_x(x^*,y^*,\lambda^*)=f'_x(x^*,y^*)+\lambda \varphi'_x(x^*,y^*) = 0 \\
\mathcal{L}'_y(x^*,y^*,\lambda^*)=f'_y(x^*,y^*)+\lambda \varphi'_y(x^*,y^*) = 0 \\
\mathcal{L}'_\lambda(x^*,y^*,\lambda^*) = \varphi'(x^*,y^*) = 0
\end{cases}}  \\ 
$\Rightarrow$ система трех уравнений с тремя неизвестными \\ $\Rightarrow$ находим стационарные точки $(x^*,y^*,\lambda^*)$ \\

\noindent Вычисляется в каждой из get-х стационарных точек $(x^*,y^*,\lambda^*)$ определитель:\\

$\Delta$ = - 
$\begin{vmatrix}
0&\varphi'_x&\varphi'_y \\
\varphi'_x& \mathcal{L}''_{xx}&\mathcal{L}''_{xy} \\
\varphi'_y& \mathcal{L}''_{yx}&\mathcal{L}''_{yy}
\end{vmatrix}$ \\

\noindent if $\Delta > 0 \Rightarrow$ then т.$(x^*,y^*,\lambda^*) \in$ locmin z \\
if $\Delta < 0 \Rightarrow$ then т.$(x^*,y^*,\lambda^*) \in$ locmax z

\subsection{Условная оптимизация с ограничениями-равенствами и неравенствами. Математическое программирование}
\textbf{def:} Задачей условной оптимизации с ограничениями-равенствами и ограничениями-неравентсвами называется следующая задача: \\
(4)$f_0(x) \rightarrow min$; (5)$f_i(x) \leq 0$ $(i = \overline{1,p})$, (6)$f_i(x) = 0$ $(i = \overline{p+1,m})$, \\ 
где $f_i(x): R^n \rightarrow R (i = \overline{0,m})$ \\

\noindent \textbf{def:} Эта задача называется задачей математического программирования (ЗМП) \\
\noindent \textbf{Rem:} \\
1) Обычно в ЗМП присутствуют условия неотрицательности переменных $x_i \geq (i = \overline{1,n})$ эти условия записываются в системе неравенств (5) в виде: $-x_i \leq 0 (i = \overline{1,n})$ \\
2) Каждое ограничение-равенство (6) можно заменить двумя неравенствами: \\
\eqn{f_i(x) = 0 \Leftrightarrow 
\begin{cases}
f_i(x) \leq 0 \\
-f_i(x) \leq 0 
\end{cases} (i = \overline{p+1, m})}

\noindent В силу этих замечаний ЗМП можно записать в виде: \\
$f_0(x) \rightarrow min$ \\
$f_i(x) \leq 0$ $(i = \overline{1,m})$ (8)
где $x = (x_1,...,x_n)$

Для ЗМП составляется функция Лагранжа
\begin{equation*}
\mathcal{L}(x_1,...,x_n;\lambda_1,...,\lambda_m) = f_0(x1,...,x_n) + \sum_{i=1}^{m} \lambda_i f_i(x_1,...,x_n)
\end{equation*} 
С помощью функции Лагранжа выписываются необходимые и достаточные условия экстремума тип (2)-(3). Однако,  проверка выполнения этих условий становится еще более сложной. При этом требуется решать систему, вообще говоря, нелинейных уравнений и неравенств. Для этого применяются итерационные численные методы, формирующие последовательность точек, сходящуюся к точке экстремума. Однако, эта точка может оказаться точкой локального(а не глобального) экстремума. Это объясняется тем, что ЗМП в такой общей постановке, без каких-либо предположений относительной функций $f_i(x)$, является многоэкстремальной задачей. Не существует универсальных методов решения таких задач. Содержательная теория построена лишь для отдельных классов ЗМП, в частности, задач оптимизации выпуклых функций на выпуклом множестве решений систем ограничений-неравенств. Такие задачи, называемые задачами выпуклого программирования(ЗВП), являются, как будет показано ниже, одноэкстремальными задачами.
