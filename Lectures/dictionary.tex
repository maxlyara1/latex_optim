\ProvidesFile{lecture09.tex}[Лекция 9]

\newpage

\section{Словарик выражений}
\begin{enumerate}
    \item get-уем --- получаем
    \item $\div$ --- соответствует
    \item 've --- имеем
    \item зча --- задача
    \item зчи ---  задачи
    \item зч --- задач
    \item Мцы --- матрицы
    \item if --- если
    \item Т.О. --- таким образом
    \item opt-ные --- оптимальные
    \item then --- тогда
    \item смы --- системы
    \item М. --- можно
    \item max-я --- максимизация
    \item min-я --- минимизация
    \item реш-я --- решения
    \item def-уем --- определяем
    \item def-ся --- определяется
    \item def-ть --- определить
    \item the end --- конец
    \item $\forall$ --- для всех
    \item $\exists$ --- существует
    \item only --- только
    \item \qedsymbol --- что и требовалось доказать
    \item Rem --- замечание
    \item т.н. --- так называется
    \item f.e. --- for example (например)
\end{enumerate}